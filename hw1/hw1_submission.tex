%%%%%START-of-DOCUMENT%%%%%%%%%%%%%%%
\documentclass[12pt]{article}
\usepackage{amsmath,amssymb,amsthm}
\usepackage{hyperref}
\usepackage{graphicx}
\usepackage[margin=1in]{geometry}
\usepackage{fancyhdr}
\usepackage{mathtools}
\DeclarePairedDelimiter\ceil{\lceil}{\rceil}
\DeclarePairedDelimiter\floor{\lfloor}{\rfloor}
\setlength{\parindent}{0pt}
\setlength{\parskip}{5pt plus 1pt}
\setlength{\headheight}{13.6pt}

\newcommand\question[3]{\vspace{.25in}\textbf{#1: #2}\vspace{.5em}\hrule\vspace{.10in}}
\renewcommand\part[1]{\vspace{.10in}\textbf{(#1)}}
\newcommand\algorithm{\vspace{.10in}\textbf{Complexity: }}
\newcommand\correctness{\vspace{.10in}\textbf{Correctness: }}
\newcommand\runtime{\vspace{.10in}\textbf{Running time: }}
\pagestyle{fancyplain}
\lhead{{\NAME\ (\ANDREWID)}}
%\chead{\textbf{HW\HWNUM}}
\rhead{\today}

\begin{document}\raggedright
\title{CS 575\\Homework 1}
\date{}
\maketitle
“I have done this assignment completely on my own. I have not
copied it, nor have I given my solution to anyone else. I under-
stand that if I am involved in plagiarism or cheating I will have
to sign an official form that I have cheated and that this form will
be stored in my official university record. I also understand that
I will receive a grade of 0 for the involved assignment for my first
offense and that I will receive a grade of “F” for the course for any
additional offense.”
\begin{flushright}
Aniruddha Tekade
\end{flushright}
\hrulefill
%Section A==============Change the values below to match your information==================
\newcommand\NAME{Aniruddha Tekade}  	% your name
\newcommand\ANDREWID{B00618834}     	% your andrew id
\newcommand\HWNUM{1}              	% the homework number
%Section B==============Put your answers to the questions below here=======================
% no need to restate the problem --- the graders know which problem is which, but replacing "The First Problem" with a short phrase will help you remember which problem this is when you read over your homeworks to study.

\question{1}{Prove the following using the original definition of O, $\Omega$, $\Theta$, $o$, $\omega$} 

\part{a} \textbf{10$n^3$ + 2n + 15 = $O$($n^3$)}
\\
\textbf{Explation $\Rightarrow$} Definition of \texttt{Big-Oh} is 0 $\leq$ g(n) $\leq$ cf(n) for all n $\geq$ N 
\\
Applying this definition to the given statement,  we get -
\\
\hspace{3cm}\textbf{0 $\leq$ 10$n^3$ + 2n + 15 $\leq$ c$n^3$} 
\\
That means there must exists c$>$0	and	integer	N$>$0.
\\Dividing both sides of the inequality by	$n^3$$>$0 we get:
\\ \hspace{3cm}\textbf{0 $\leq$ 10 + 2/$n^2$ + 15/$n^3$ $\leq$ c for all n$\geq$N}
\\
(2/$n^2$ + 15/$n^3$)	 $>$ 0 becomes smaller as n increases
\\
Clearly any c can be chosen anything $\geq$ 27 if we put N = 1. 
\\ For instance c = 27, N = 1
\\Hence, 10$n^3$ + 2n + 15 = $O$($n^3$)

%% + 15 $\leq$ $O$($n^3$)}

\part{b} \textbf{7$n^2$ = $\Omega$(n)}
\\
\textbf{Explation $\Rightarrow$} Definition of \texttt{Big-Omega} is 0 $\leq$ cf(n) $\leq$ g(n) for all n $\geq$ N 
\\
Applying this definition to the given statement,  we get -
\\
\hspace{3cm}\textbf{0 $\leq$ c$n$ $\leq$ 7$n^2$} 
\\
That means there must exists c$>$0	and	integer	N$>$0.
\\Dividing both sides of the inequality by	$n$$>$0 we get:
\\ \hspace{3cm}\textbf{0 $\leq$ c $\leq$ 7$n$ for all n$\geq$N}
\\ Clearly, there are many choices for c and N. 
\\ For instance c = 7, N = 1
\\ Hence, 7$n^2$ = $\Omega$(n)

\part{c} \textbf{5$n^2$ = $\omega$(n)}
\\
\textbf{Explation $\Rightarrow$} Definition of \texttt{small-0mega} is 0 $\leq$ cf(n) $\leq$ g(n) for all n $\geq$ N, where cf(n) is strictly less than g(n). 
\\
Applying this definition to the given statement,  we get -
\\
\hspace{3cm}\textbf{0 $\leq$ c$n$ $\leq$ 5$n^2$} 
\\
That means there must exists c$>$0	and	integer	N$>$0.
\\Dividing both sides of the inequality by	$n$$>$0 we get:
\\ \hspace{3cm}\textbf{0 $\leq$ c $\leq$ 5n for all n$\geq$N}
\\ Clearly, there are many choices for c and N. 
\\ For instance c = 5, N = 1
\\ Hence, 5$n^2$ = $\omega$(n)

\part{d} \textbf{7$n^3$ + 15$n^2$ + 5 = $\Theta$($n^3$)}
\\
\textbf{Explation $\Rightarrow$} To prove 7$n^3$ + 15$n^2$ + 5 = $\Theta$($n^3$) we nee to first prove that 7$n^3$ + 15$n^2$ + 5 = $O$($n^3$) as well as 7$n^3$ + 15$n^2$ + 5 = $\Omega$($n^3$)
\newline
\textbf{Proof that prove that 7$n^3$ + 15$n^2$ + 5 = $O$($n^3$) $\Rightarrow$}
\newline
Acording to the definition of \texttt{Big-Oh} is 0 $\leq$ 7$n^3$ + 15$n^2$ + 5 $\leq$ c$n^3$ for all n $\geq$ N.
\newline
That means there must exists c$>$0	and	integer	N$>$0.
\\Dividing both sides of the inequality by	$n^3$$>$0 we get:
\\ \hspace{3cm}\textbf{0 $\leq$ 7 + 15$n^2$ + 5/$n^3$ $\leq$ c for all n$\geq$N}
\\(15$n^2$ + 5/$n^3$)	$>$ 0 becomes smaller as n increases
\\
Clearly any c can be chosen anything $\geq$ 27 if we put N = 1. 
\\ For instance c = 27, N = 1
\\Hence, \textbf{7$n^3$ + 15$n^2$ + 5 = $O$($n^3$)}

\textbf{Proof that prove that 7$n^3$ + 15$n^2$ + 5 = $\Omega$($n^3$) $\Rightarrow$}
\newline
Acording to the definition of \texttt{Big-Omega} is 0 $\leq$ 7$n^3$ + 15$n^2$ + 5 $\leq$ c$n^3$ for all n $\geq$ N.
\newline
That means there must exists c$>$0	and	integer	N$>$0.
\\Dividing both sides of the inequality by	$n^3$$>$0 we get:
\\ \hspace{3cm}\textbf{0 $\leq$ c$n^3$ $\leq$ 7 + 15$n^2$ + 5/$n^3$ for all n$\geq$N}
\\(15$n^2$ + 5/$n^3$)	$>$ 0 becomes greater as n grows 
\\ For instance c = 27, N = 1
\\Hence, \textbf{7$n^3$ + 15$n^2$ + 5 = $O$($n^3$)}
\newline 
Since both Big-Oh and Big-Omega complexities satisfies, it means, there exists, asymptotically Theta bound for this relationship. Hence we can conclude that - \textbf{7$n^3$ + 15$n^2$ + 5 = $\Theta$($n^3$)}

%===============================================================================
\part{e} \textbf{p(n) = $\sum_{i=1}^{k}$ $a^i$$n^i$ = $\Theta$($n^k$); where $a_{i}$ $>$ 0}
\\
\textbf{Explation $\Rightarrow$} \\Let's prove that \textbf{$\sum_{i=1}^{k}$ $a^i$$n^i$ = $O$($n^k$)}
\\ By definition of Big-Oh, we have - \\
$\Rightarrow$ 0 $\leq$ $\sum_{i=1}^{k}$ $a^i$$n^i$ $\leq$ c$n^k$ \\
$\Rightarrow$  $\sum_{i=1}^{k}$ $a^i$ $\sum_{i=1}^{k}$$n^i$ $\leq$ c$n^k$ \\
$\Rightarrow$ We know that - $\sum_{i=1}^{k}$ $a^i$ = 1 + a + $a^2$ + $a^3$ + ... + $a^k$ = $\frac{a^{k+1}-1}{a-1}$ \& \\
$\Rightarrow$ $\sum_{i=1}^{k}$ $n^i$ = 1 + n + $n^2$ + $n^3$ + ... + $n^k$ = $\frac{n^{k+1}-1}{n-1}$ \newline \\
Combining above two results we get - \newline \\
($\frac{a^{k+1}-1}{a-1}$) ($\frac{n^{k+1}-1}{n-1}$) $\leq$ c$n^k$
$\Rightarrow$ which clears that there is always a constant c $>$ 0. \newline \\Therefore, $\sum_{i=1}^{k}$ $a^i$$n^i$ = $O$($n^k$)
\newline \\ Similarly we can prove the for $\Omega$, that $\sum_{i=1}^{k}$ $a^i$$n^i$ = $\Omega$($n^k$)

Therefore, p(n) = $\sum_{i=1}^{k}$ $a^i$$n^i$ = $\Theta$($n^k$); where $a_{i}$ $>$ 0
%================================================================================

\question{2}{Prove the following using limits.}

\part{a} \textbf{$n^k$ = $o$($3^n$); where k $>$ 0}
\\
\textbf{Explanation $\Rightarrow$} $n^k$ $\in$ $o$($2^n$) where k is a positive integer.
\\
$3^n$ = $e^{nln3}$
\\
($3^n$)' = ($e^{nln3}$)' = ln2$e^{nln3}$ = ln3($3^n$) = $\lim_{n\to\infty} \frac{kn^(k-1)}{3^nln2}$
\\
= $\lim_{n\to\infty} \frac{n^k}{3^n}$ = $\lim_{n\to\infty} \frac{kn^(k-1)}{3^nln2}$
\newline \\
 = $\lim_{n\to\infty} \frac{k(k-1)n^{k-2}}{3^nln^22}$ = ... = $\lim_{n\to\infty} \frac{k!}{3^nln^k2}$
\newline \\
Therefore, $n^k$ = $o$($3^n$); where k $>$ 0

\part{b} \textbf{n = $\omega$(lg $n^5$)}
\\
\textbf{Explanation $\Rightarrow$} We know that lg $n^5$ = $\frac{ln n^5}{ln 2}$
\newline \\
Taking derivative,  we get - (lg $n^5$)' = ($\frac{ln n^5}{ln 2}$)' = $\frac{5}{n ln 2}$ 
\newline \\
$\lim_{n\to\infty} \frac{n}{lg n^5}$ = $\lim_{n\to\infty} \frac{n'}{(lg n^5)'}$
\newline
\\ 
$\lim_{n\to\infty} \frac{n lg 2}{5}$
\newline
\\ Therefore, n is greater than lg $n^5$. We can say, n = $\omega$(lg $n^5$)

%==================================================================

\question{3} {Prove that $3^n$-1 is divisible by 2 for n = 1, 2, 3, ... by induction. Divide your proof into the three required parts: Induction Base,
Induction Hypothesis, and Induction Steps.}

\textbf{Base Case} $\Rightarrow$ If n = 1, $3^n$-1 is divisible by 2. 
\newline $3^n$-1 = $3^1$-1 = 3 - 1 = 2. \& 2 is divisible by 2.
\newline \\
\textbf{Assumption} $\Rightarrow$ It is true that $3^k$-1 is also divisible by 2 where k is any random integer that is greater than 0.
\newline \\ 
\textbf{Proof} $\Rightarrow$ \newline
If we put k = 2 $\rightarrow$ $3^2$-1 = 9 - 1 = 8. 8\%2=0 \\
If we put k = 3 $\rightarrow$ $3^3$-1 = 27 - 1 = 26. 26\%2=0 \\
If we put k = 4 $\rightarrow$ $3^4$-1 = 81 - 1 = 80. 80\%2=0 \\
If we put k = 5 $\rightarrow$ $3^5$-1 = 243 - 1 = 242. 242\%2=0
\newline \\
Therefore it is true that, $3^n$-1 is divisible by 2 for any integer value of n.	 
%=================================================================

\question{4} {Prove or disprove that $n^3$ = O($n^2$)}

\textbf{Explanation} $\Rightarrow$ Lets apply the definition of \texttt{Big-Oh} on above question - \\
$\Rightarrow$ 0 $\leq$ $n^3$ $\leq$ c$n^2$
$\Rightarrow$ $n^3$ $\leq$ c$n^2$
$\Rightarrow$ Let's divide both sides by $n^2$
$\Rightarrow$ n $\leq$ c
$\Rightarrow$ Above expression states that, n will always be smaller than a constant c, which is absolutely false since as n grows, c being a constant become small at some point or the other.
$\Rightarrow$ Therefore it is proved that $n^3$ $\neq$ O($n^2$)
%=============================================================================================

\question{5} {Just say True or False for the following}

\part{a} 1000000$n^2$ + 5000 = $\Theta$ ($n^2$) $\Rightarrow$ \textbf{True}

\part{b} $2^{n+1}$ = O($2^n$) $\Rightarrow$ \textbf{True}

\part{c} $n^3$ + $n^2$ + 100n = $\Omega$($n^3$) $\Rightarrow$ \textbf{True}

\part{d} $n^{1000}$ = $\omega$($2^n$) $\Rightarrow$ \textbf{False}

\part{e} $\log$ $n^{100}$ = $\Omega$($\log$ n) $\Rightarrow$ \textbf{True}

\question{6} {Analyze the worst case time complexity of recursive binary search
using the iterative method. Assume the number of data n = $2^k$}

\textbf{Explanation} $\Rightarrow$ The recurrence equation for binary search is as follows -

T(1) = 1; for  \\
T(n) = 1+T($\lfloor$ n/2 $\rfloor$); for n$>$2

Applying recurrence oprations - \\
at \textit{i=1} $\Rightarrow$ T(n) = 1 + T($\lfloor$ n/2 $\rfloor$) \\
at \textit{i=2} $\Rightarrow$ T(n/2) = 2 + T($\lfloor$ n/4 $\rfloor$) \\
at \textit{i=3} $\Rightarrow$ T(n/4) = 3 + T($\lfloor$ n/8 $\rfloor$) \\
. . .\\
. . .\\
. . .\\
at \textit{i=n-1} $\Rightarrow$ T(n/$2^k$) = K+T(1) \newline
\\ Terminating the recurrence, we get, \\
$\Rightarrow$ T(n) = k + T($\lfloor$1$\rfloor$) = ($\lfloor$log n$\rfloor$)+1 \\
$\Rightarrow$ We know that, data is n = $2^k$, which means k = $\log$ n \\
$\Rightarrow$ Therefore, we can prove that the worst case time complexity of binary search method is $O$($\log$n)
%===========================================================================================

\question{7} {The following pseudo code computes a factorial for input parameter n that is expressed via s bits where n, s $\geq$ 1. What is the time
complexity of the pseudo code?}

\begin{verbatim}
unsigned int fact(unsigned int n) {
unsigned int p = 1;
for (i=1; i≤ n; i++) p = p * i;
return p;
}
\end{verbatim}
\textbf{Explanation} $\Rightarrow$ The analysis of factorial algorithms is as follows - 
\begin{itemize}
	\item Total number of operations = $n$
	\item Number bits for value $n$ is $s$ = $\lfloor$($\log$ x)$\rfloor$+1
	\item $\lfloor$($\log$ x)$\rfloor$ = $s$ - 1
	\item $n$ $\geq$ $2^{s-1}$
	\item Therefore, there are \texttt{Exponential} number of operations in terms of $s$
\end{itemize}  
\end{document}
